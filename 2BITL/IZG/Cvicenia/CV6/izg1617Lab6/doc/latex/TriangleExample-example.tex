\hypertarget{TriangleExample-example}{\section{Triangle\-Example}
}
\hyperlink{triangleExample_8c}{triangle\-Example.\-c}  \hypertarget{triangleExample.c_Globals}{}\subsection{Globální proměnné}\label{triangleExample.c_Globals}
Globální proměnné jsou uloženy ve struktuře\-: 
\begin{DoxyCodeInclude}

\textcolor{keyword}{struct }\hyperlink{structTriangleExampleVariables}{TriangleExampleVariables}\{
  GLuint \hyperlink{structTriangleExampleVariables_abc287e489a25d4e3c4ad1899d183881d}{program};\textcolor{comment}{//program id}
\textcolor{comment}{}  GLint  \hyperlink{structTriangleExampleVariables_a0215b92b36815394d6a4c8b5e3baa27a}{projectionMatrixUniform};\textcolor{comment}{//location}
\textcolor{comment}{}  GLint  \hyperlink{structTriangleExampleVariables_a94cf55d21d35fb4d0f85c1d7a49e6474}{viewMatrixUniform};\textcolor{comment}{//location}
\textcolor{comment}{}  GLuint \hyperlink{structTriangleExampleVariables_a4230de13079947c4093b8d75ad1d5035}{vao};\textcolor{comment}{//vertex array id}
\textcolor{comment}{}  GLuint \hyperlink{structTriangleExampleVariables_af3b747228ed4a26fffca56a69838ecae}{vbo};\textcolor{comment}{//vertex buffer object id}
\textcolor{comment}{}  GLuint \hyperlink{structTriangleExampleVariables_a9e7c6acc784faacf21bca7c46f9f4244}{ebo};\textcolor{comment}{//vertex indices}
\}\hyperlink{triangleExample_8c_af82b723635ac0c90962571915a1b1163}{triangleExample};
\end{DoxyCodeInclude}
Projekční a view matice jsou uloženy v externích proměnných\-: 
\begin{DoxyCodeInclude}
\textcolor{keyword}{extern} \hyperlink{structMat4}{Mat4} \hyperlink{mouseCamera_8c_a1f344d924f733a22d7659db612a0efe8}{projectionMatrix};\textcolor{comment}{//projection matrix}
\textcolor{comment}{}\textcolor{keyword}{extern} \hyperlink{structMat4}{Mat4} \hyperlink{mouseCamera_8c_ae64ebe4c77936fc93d161b97bd8e96df}{viewMatrix}      ;\textcolor{comment}{//view matrix}
\end{DoxyCodeInclude}
Pokud se hýbe s myší, jsou tyto matice přepočítávány.\par
 Zdrojáky verttex a fragment shaderu jsou uloženy v proměnných\-: 
\begin{DoxyCodeInclude}
\textcolor{keywordtype}{char} \textcolor{keyword}{const}* \hyperlink{triangleExample_8c_a90e98a213f90e03bd807c2f64031a8e2}{triangleExampleVertexShaderSource} = 
\textcolor{stringliteral}{"#version 330\(\backslash\)n"}
\textcolor{stringliteral}{"uniform mat4 projectionMatrix;\(\backslash\)n"}
\textcolor{stringliteral}{"uniform mat4 viewMatrix;\(\backslash\)n"}
\textcolor{stringliteral}{"layout(location=0)in vec3 position;\(\backslash\)n"}
\textcolor{stringliteral}{"out vec3 vColor;"}
\textcolor{stringliteral}{"void main()\{\(\backslash\)n"}
\textcolor{stringliteral}{"  gl\_Position = projectionMatrix*viewMatrix*vec4(position,1);\(\backslash\)n"}
\textcolor{stringliteral}{"  if(gl\_VertexID == 0)vColor = vec3(1.f,0.f,0.f);\(\backslash\)n"}
\textcolor{stringliteral}{"  if(gl\_VertexID == 1)vColor = vec3(0.f,1.f,0.f);\(\backslash\)n"}
\textcolor{stringliteral}{"  if(gl\_VertexID == 2)vColor = vec3(0.f,0.f,1.f);\(\backslash\)n"}
\textcolor{stringliteral}{"  if(gl\_VertexID == 3)vColor = vec3(0.f,1.f,1.f);\(\backslash\)n"}
\textcolor{stringliteral}{"\}\(\backslash\)n"};
\end{DoxyCodeInclude}

\begin{DoxyCodeInclude}
\textcolor{keywordtype}{char} \textcolor{keyword}{const}* \hyperlink{triangleExample_8c_a48b175665aab439caedaa0440a85462e}{triangleExampleFragmentShaderSource} = 
\textcolor{stringliteral}{"#version 330\(\backslash\)n"}
\textcolor{stringliteral}{"in vec3 vColor;\(\backslash\)n"}
\textcolor{stringliteral}{"layout(location=0)out vec4 fColor;\(\backslash\)n"}
\textcolor{stringliteral}{"void main()\{\(\backslash\)n"}
\textcolor{stringliteral}{"  fColor = vec4(vColor,1.f);"}
\textcolor{stringliteral}{"\}\(\backslash\)n"};
\end{DoxyCodeInclude}
\hypertarget{triangleExample.c_Initialization}{}\subsection{Inicializace}\label{triangleExample.c_Initialization}
Když je příklad spuštěn, je zavolána funkce \hyperlink{triangleExample_8c_a73184b4ab6bb513ad9a9a4c36e92646b}{triangle\-Example\-\_\-on\-Init()}. Tato funkce vytvoří opengl objekty (buffer, vertex arrays, shader programy). První funkční volání inicializuje matice\-: 
\begin{DoxyCodeInclude}
  \hyperlink{mouseCamera_8c_a7e7e918a9328502b7c35cfbbdb068b7b}{cpu\_initMatrices}(width,height);
\end{DoxyCodeInclude}
\hypertarget{triangleExample.c_ShaderProgram}{}\subsubsection{Inicializace shader programu}\label{triangleExample.c_ShaderProgram}
Dále následuje kopilace shaderů a linkování shader programu\-: 
\begin{DoxyCodeInclude}
  GLuint \textcolor{keyword}{const} vertexId = \hyperlink{program_8h_aeeb65abe90cc1be97e5788afe8ca57a7}{compileShader}(
      GL\_VERTEX\_SHADER                 , \textcolor{comment}{//a type of shader}
      triangleExampleVertexShaderSource);\textcolor{comment}{//a glsl source}

  GLuint \textcolor{keyword}{const} fragmentId = \hyperlink{program_8h_aeeb65abe90cc1be97e5788afe8ca57a7}{compileShader}(
      GL\_FRAGMENT\_SHADER                 , \textcolor{comment}{//a type of shader}
      triangleExampleFragmentShaderSource);\textcolor{comment}{//a glsl source}

  \hyperlink{triangleExample_8c_af82b723635ac0c90962571915a1b1163}{triangleExample}.\hyperlink{structTriangleExampleVariables_abc287e489a25d4e3c4ad1899d183881d}{program} = \hyperlink{program_8h_af917a75fb9e573fb52d85ef90f32231e}{linkProgram}(vertexId,fragmentId);
\end{DoxyCodeInclude}
Následuje získaní lokací (id) uniformních proměnných v shader programu pro matice\-: 
\begin{DoxyCodeInclude}
  \hyperlink{triangleExample_8c_af82b723635ac0c90962571915a1b1163}{triangleExample}.\hyperlink{structTriangleExampleVariables_a0215b92b36815394d6a4c8b5e3baa27a}{projectionMatrixUniform} = glGetUniformLocation(
      \hyperlink{triangleExample_8c_af82b723635ac0c90962571915a1b1163}{triangleExample}.\hyperlink{structTriangleExampleVariables_abc287e489a25d4e3c4ad1899d183881d}{program}, \textcolor{comment}{//a program id}
      \textcolor{stringliteral}{"projectionMatrix"}     );\textcolor{comment}{//a uniform variable name}

  \hyperlink{triangleExample_8c_af82b723635ac0c90962571915a1b1163}{triangleExample}.\hyperlink{structTriangleExampleVariables_a94cf55d21d35fb4d0f85c1d7a49e6474}{viewMatrixUniform} = glGetUniformLocation(
      \hyperlink{triangleExample_8c_af82b723635ac0c90962571915a1b1163}{triangleExample}.\hyperlink{structTriangleExampleVariables_abc287e489a25d4e3c4ad1899d183881d}{program}, \textcolor{comment}{//a program id}
      \textcolor{stringliteral}{"viewMatrix"}           );\textcolor{comment}{//a uniform variable name}
\end{DoxyCodeInclude}
\hypertarget{triangleExample.c_Buffers}{}\subsubsection{Inicializace bufferů}\label{triangleExample.c_Buffers}
Následuje inicializace bufferu pro vrcholy. Vrcholy jsou uloženy ve statickém poli\-: 
\begin{DoxyCodeInclude}
  \textcolor{keywordtype}{float} \textcolor{keyword}{const} positions[12] = \{\textcolor{comment}{//vertex positions}
    -1.f,-1.f,+0.f,\textcolor{comment}{//a quad vertex A}
    +1.f,-1.f,+0.f,\textcolor{comment}{//a quad vertex B}
    -1.f,+1.f,+0.f,\textcolor{comment}{//a quad vertex C}
    +1.f,+1.f,+0.f,\textcolor{comment}{//a quad vertex D}
  \};
\end{DoxyCodeInclude}
Nejprve je nutné zarezervovat jmeno(id) bufferu\-: 
\begin{DoxyCodeInclude}
  glGenBuffers(
      1                   , \textcolor{comment}{//a number of buffer ids that will be reserved}
      &\hyperlink{triangleExample_8c_af82b723635ac0c90962571915a1b1163}{triangleExample}.\hyperlink{structTriangleExampleVariables_af3b747228ed4a26fffca56a69838ecae}{vbo});\textcolor{comment}{//a pointer to buffer id variable}
\end{DoxyCodeInclude}
Poté je nutné id bufferu navázat na vhodný binding point opengl\-: 
\begin{DoxyCodeInclude}
  glBindBuffer(
      GL\_ARRAY\_BUFFER    , \textcolor{comment}{//a binding point}
      \hyperlink{triangleExample_8c_af82b723635ac0c90962571915a1b1163}{triangleExample}.\hyperlink{structTriangleExampleVariables_af3b747228ed4a26fffca56a69838ecae}{vbo});\textcolor{comment}{//a buffer id}
\end{DoxyCodeInclude}
G\-L\-\_\-\-A\-R\-R\-A\-Y\-\_\-\-B\-U\-F\-F\-E\-R binding point slouží pro vertex attributy. Následuje nahrání dat na G\-P\-U\-: 
\begin{DoxyCodeInclude}
  glBufferData(
      GL\_ARRAY\_BUFFER , \textcolor{comment}{//a binding point}
      \textcolor{keyword}{sizeof}(\textcolor{keywordtype}{float})*12, \textcolor{comment}{//size of buffer}
      positions       , \textcolor{comment}{//a pointer to data}
      GL\_STATIC\_DRAW  );\textcolor{comment}{//a hint}
\end{DoxyCodeInclude}
Následuje vytvoření bufferu pro indexy. Postup je obdobný jako pro vertex atributy, jen binding point je jiný\-: 
\begin{DoxyCodeInclude}
  uint32\_t \textcolor{keyword}{const} indices[6] = \{
    0,1,2,
    2,1,3,
  \};
  glGenBuffers(1,&\hyperlink{triangleExample_8c_af82b723635ac0c90962571915a1b1163}{triangleExample}.\hyperlink{structTriangleExampleVariables_a9e7c6acc784faacf21bca7c46f9f4244}{ebo});
  glBindBuffer(GL\_ELEMENT\_ARRAY\_BUFFER,\hyperlink{triangleExample_8c_af82b723635ac0c90962571915a1b1163}{triangleExample}.\hyperlink{structTriangleExampleVariables_a9e7c6acc784faacf21bca7c46f9f4244}{ebo});
  glBufferData(GL\_ELEMENT\_ARRAY\_BUFFER,\textcolor{keyword}{sizeof}(uint32\_t)*6,indices,GL\_STATIC\_DRAW);
\end{DoxyCodeInclude}
Tím je inicializace bufferů hotová. \hypertarget{triangleExample.c_VAO}{}\subsubsection{Inicializace vertex arrays object (nastavení vertex pulleru)}\label{triangleExample.c_VAO}
Následuje vytvoření a nastavení objektu Vertex Arrays. Tento objekt obsahuje nastavení vertex puller -\/ ten sestavuje vertexy, které přícházení do vertex shaderu. Vertex je složen z několik vertex attributů, (v tomto příkladě pouze z jednoho) Nejprve je nutné získát lokaci vstupní proměnné ve vertex shaderu, pro kterou sa nastavuje čtecí hlava\-: 
\begin{DoxyCodeInclude}
  GLint \textcolor{keyword}{const} positionAttribute = glGetAttribLocation(
      \hyperlink{triangleExample_8c_af82b723635ac0c90962571915a1b1163}{triangleExample}.\hyperlink{structTriangleExampleVariables_abc287e489a25d4e3c4ad1899d183881d}{program}, \textcolor{comment}{//a program id}
      \textcolor{stringliteral}{"position"}             );\textcolor{comment}{//a name of vertex attribute}
\end{DoxyCodeInclude}
První krok spočívá v rezervování jména (id) vertex arrays objektu (vao)\-: 
\begin{DoxyCodeInclude}
  glGenVertexArrays(
      1                   , \textcolor{comment}{//a number of vertex arrays}
      &\hyperlink{triangleExample_8c_af82b723635ac0c90962571915a1b1163}{triangleExample}.\hyperlink{structTriangleExampleVariables_a4230de13079947c4093b8d75ad1d5035}{vao});\textcolor{comment}{//a pointer to first }
\end{DoxyCodeInclude}
Nastavení vao se provádí mezi příkazi pro aktivaci a dekativaci. Další kro je proto aktivace\-: 
\begin{DoxyCodeInclude}
  glBindVertexArray(\hyperlink{triangleExample_8c_af82b723635ac0c90962571915a1b1163}{triangleExample}.\hyperlink{structTriangleExampleVariables_a4230de13079947c4093b8d75ad1d5035}{vao});
\end{DoxyCodeInclude}
Pro nastavení indexování je nutné navázat element buffer\-: 
\begin{DoxyCodeInclude}
  glBindBuffer(GL\_ELEMENT\_ARRAY\_BUFFER,\hyperlink{triangleExample_8c_af82b723635ac0c90962571915a1b1163}{triangleExample}.\hyperlink{structTriangleExampleVariables_a9e7c6acc784faacf21bca7c46f9f4244}{ebo});
\end{DoxyCodeInclude}
Pro nastavení vertex attributů je pořeba specifikovat, ze kterého bufferu bude daný atribut čten\-: 
\begin{DoxyCodeInclude}
  glBindBuffer(
      GL\_ARRAY\_BUFFER,
      \hyperlink{triangleExample_8c_af82b723635ac0c90962571915a1b1163}{triangleExample}.\hyperlink{structTriangleExampleVariables_af3b747228ed4a26fffca56a69838ecae}{vbo});
\end{DoxyCodeInclude}
Dále je nutné specifikovat, kde se vertex attribut v bufferu nachází (konfigurace čtecí hlavy)\-: 
\begin{DoxyCodeInclude}
  glVertexAttribPointer(
      (GLuint)positionAttribute, \textcolor{comment}{//an attribute index}
      3                        , \textcolor{comment}{//a number of components}
      GL\_FLOAT                 , \textcolor{comment}{//a type of attribute}
      GL\_FALSE                 , \textcolor{comment}{//a normalization}
      \textcolor{keyword}{sizeof}(\textcolor{keywordtype}{float})*3          , \textcolor{comment}{//a stride in bytes}
      (GLvoid \textcolor{keyword}{const}*)0         );\textcolor{comment}{//an offset in bytes}
\end{DoxyCodeInclude}
Vertex attribut (pozice) je složen ze 3 komponent, typu float, je uložen těsně za sebou a začíná na začátku bufferu (nulový offset). Poté je nutné vertex attribut povolit\-: 
\begin{DoxyCodeInclude}
  glEnableVertexAttribArray((GLuint)positionAttribute);
\end{DoxyCodeInclude}
Nakonec je nutné deaktivovat vao\-: 
\begin{DoxyCodeInclude}
  glBindVertexArray(0);
\end{DoxyCodeInclude}
Tím je inicializace hotová. \hypertarget{triangleExample.c_Deinit}{}\subsection{Deinicializace}\label{triangleExample.c_Deinit}
Deinicializace/uvolnění zdrojů probíhá ve funkci \hyperlink{triangleExample_8c_a35b10f59fe16423447777ba6bbba3b66}{triangle\-Example\-\_\-on\-Exit()}. V této funkci je nutné uvolnit program bufffery a vertex arrays object\-: 
\begin{DoxyCodeInclude}
  glDeleteProgram(\hyperlink{triangleExample_8c_af82b723635ac0c90962571915a1b1163}{triangleExample}.\hyperlink{structTriangleExampleVariables_abc287e489a25d4e3c4ad1899d183881d}{program});
  glDeleteBuffers(1,&\hyperlink{triangleExample_8c_af82b723635ac0c90962571915a1b1163}{triangleExample}.\hyperlink{structTriangleExampleVariables_af3b747228ed4a26fffca56a69838ecae}{vbo});
  glDeleteBuffers(1,&\hyperlink{triangleExample_8c_af82b723635ac0c90962571915a1b1163}{triangleExample}.\hyperlink{structTriangleExampleVariables_a9e7c6acc784faacf21bca7c46f9f4244}{ebo});
  glDeleteVertexArrays(1,&\hyperlink{triangleExample_8c_af82b723635ac0c90962571915a1b1163}{triangleExample}.\hyperlink{structTriangleExampleVariables_a4230de13079947c4093b8d75ad1d5035}{vao});
\end{DoxyCodeInclude}
\hypertarget{triangleExample.c_Shaders}{}\subsection{Shadery}\label{triangleExample.c_Shaders}
Shadery jsou uloženy ve statických řetězcích. Shadery jsou psány v jazyce G\-L\-S\-L. Vertex shader v tomto příkladě promítá vrcholy pomocí matic do clip-\/space a počítá barvu vrcholu z jeho pořadového čísla. Každý shader musí mít na prvním řádku uvedenou verzi. 
\begin{DoxyCodeInclude}
\textcolor{stringliteral}{"#version 330\(\backslash\)n"}
\end{DoxyCodeInclude}
\hypertarget{triangleExample.c_VertexShader}{}\subsubsection{Vertex Shader}\label{triangleExample.c_VertexShader}
Vertex shader začíná deklarací interface. Interface je složen z uniformních proměnných\-: 
\begin{DoxyCodeInclude}
\textcolor{stringliteral}{"uniform mat4 projectionMatrix;\(\backslash\)n"}
\textcolor{stringliteral}{"uniform mat4 viewMatrix;\(\backslash\)n"}
\end{DoxyCodeInclude}
Vstupního vertex attributu pro pozici vrcholu\-: 
\begin{DoxyCodeInclude}
\textcolor{stringliteral}{"layout(location=0)in vec3 position;\(\backslash\)n"}
\end{DoxyCodeInclude}
A výstupního vertex atributu pro barvu vrcholu\-: 
\begin{DoxyCodeInclude}
\textcolor{stringliteral}{"out vec3 vColor;"}
\end{DoxyCodeInclude}
Pokud před deklarací proměnné leží klíčoví slovo in, out nebo uniform je proměnná součástí interface. Z proměnných \char`\"{}in\char`\"{} a \char`\"{}uniform\char`\"{} lze pouze číst. Do proměnných \char`\"{}out\char`\"{} lze pouze zapisovat. Proměnné \char`\"{}in\char`\"{} a \char`\"{}out\char`\"{} mohou obsahovat pro každý vrchol jinou hodnotu. Proměnné \char`\"{}uniform\char`\"{} slouží pro uložení konstant, které zůstávají stejné po dobu jednoho vykreslujícího příkazu. \char`\"{}layout(location=n)\char`\"{} explicitně specifikuje lokaci dané poměnné."\par
 Tělo shaderu tvoří funkci main\-: 
\begin{DoxyCodeInclude}
\textcolor{stringliteral}{"void main()\{\(\backslash\)n"}
\end{DoxyCodeInclude}
Ve funkci main se nejprve spočítá pozice\-: 
\begin{DoxyCodeInclude}
\textcolor{stringliteral}{"  gl\_Position = projectionMatrix*viewMatrix*vec4(position,1);\(\backslash\)n"}
\end{DoxyCodeInclude}
Následně se spočíta brava z čísla vrcholu\-: 
\begin{DoxyCodeInclude}
\textcolor{stringliteral}{"  if(gl\_VertexID == 0)vColor = vec3(1.f,0.f,0.f);\(\backslash\)n"}
\textcolor{stringliteral}{"  if(gl\_VertexID == 1)vColor = vec3(0.f,1.f,0.f);\(\backslash\)n"}
\textcolor{stringliteral}{"  if(gl\_VertexID == 2)vColor = vec3(0.f,0.f,1.f);\(\backslash\)n"}
\textcolor{stringliteral}{"  if(gl\_VertexID == 3)vColor = vec3(0.f,1.f,1.f);\(\backslash\)n"}
\end{DoxyCodeInclude}
\hypertarget{triangleExample.c_FragmentShader}{}\subsubsection{Fragment Shader}\label{triangleExample.c_FragmentShader}
Stejně jako vertex shader, fragment shader také začíná deklarací interface.\par
 Interface je složen ze vstupního fragment atributu barvy\-: 
\begin{DoxyCodeInclude}
\textcolor{stringliteral}{"in vec3 vColor;\(\backslash\)n"}
\end{DoxyCodeInclude}
Název proměnné musí odpovídat názvu příslušné proměnné ve vertex shaderu, liší se pouze v kvalifikátoru \char`\"{}in/out\char`\"{}.\par
 Následuje specifikace výstupního fragment atributu\-: 
\begin{DoxyCodeInclude}
\textcolor{stringliteral}{"layout(location=0)out vec4 fColor;\(\backslash\)n"}
\end{DoxyCodeInclude}
Výstupní atribut na pozici 0 je automaticky zapsán přes per-\/fragment operace do pole barevframebufferu. Ve funkci main je barva pouze zapsána ze vstupní proměnné do výstupní proměnné\-: 
\begin{DoxyCodeInclude}
\textcolor{stringliteral}{"  fColor = vec4(vColor,1.f);"}
\end{DoxyCodeInclude}
\hypertarget{triangleExample.c_Draw}{}\subsection{Kreslení}\label{triangleExample.c_Draw}
Kreslení je zajištěno funkci \hyperlink{triangleExample_8c_a3c6a1869a7e6614856badd527a9c332f}{triangle\-Example\-\_\-on\-Draw()}. Funkce nejprve vyčistí framebuffer\-: 
\begin{DoxyCodeInclude}
  glClear(
      GL\_COLOR\_BUFFER\_BIT| \textcolor{comment}{//clear color buffer}
      GL\_DEPTH\_BUFFER\_BIT);\textcolor{comment}{//clear depth buffer}
\end{DoxyCodeInclude}
Následně aktivuje shader program\-: 
\begin{DoxyCodeInclude}
  glUseProgram(\hyperlink{triangleExample_8c_af82b723635ac0c90962571915a1b1163}{triangleExample}.\hyperlink{structTriangleExampleVariables_abc287e489a25d4e3c4ad1899d183881d}{program});
\end{DoxyCodeInclude}
Aktivuje vertex arrays object (vao)\-: 
\begin{DoxyCodeInclude}
  glBindVertexArray(\hyperlink{triangleExample_8c_af82b723635ac0c90962571915a1b1163}{triangleExample}.\hyperlink{structTriangleExampleVariables_a4230de13079947c4093b8d75ad1d5035}{vao});
\end{DoxyCodeInclude}
Nahraje aktuální matice na G\-P\-U\-: 
\begin{DoxyCodeInclude}
  glUniformMatrix4fv(
      \hyperlink{triangleExample_8c_af82b723635ac0c90962571915a1b1163}{triangleExample}.\hyperlink{structTriangleExampleVariables_a0215b92b36815394d6a4c8b5e3baa27a}{projectionMatrixUniform}, \textcolor{comment}{//location of uniform
       variable}
      1                                      , \textcolor{comment}{//number of matrices}
      GL\_FALSE                               , \textcolor{comment}{//transpose}
      (\textcolor{keywordtype}{float}*)&projectionMatrix              );\textcolor{comment}{//pointer to data}
  glUniformMatrix4fv(
      \hyperlink{triangleExample_8c_af82b723635ac0c90962571915a1b1163}{triangleExample}.\hyperlink{structTriangleExampleVariables_a94cf55d21d35fb4d0f85c1d7a49e6474}{viewMatrixUniform}, \textcolor{comment}{//location of uniform variable}
      1                                , \textcolor{comment}{//number of matrices}
      GL\_FALSE                         , \textcolor{comment}{//transpose}
      (\textcolor{keywordtype}{float}*)&viewMatrix              );\textcolor{comment}{//pointer to data}
\end{DoxyCodeInclude}
Spustí kreslení\-: 
\begin{DoxyCodeInclude}
  glDrawElements(
      GL\_TRIANGLES    , \textcolor{comment}{//primitive type}
      6               , \textcolor{comment}{//number of vertices}
      GL\_UNSIGNED\_INT , \textcolor{comment}{//type of indices}
      (GLvoid \textcolor{keyword}{const}*)0);\textcolor{comment}{//offset}
\end{DoxyCodeInclude}
A nakonec deaktivuje vao\-: 
\begin{DoxyCodeInclude}
  glBindVertexArray(0);
\end{DoxyCodeInclude}
 
\begin{DoxyCodeInclude}
\end{DoxyCodeInclude}
 